% Options for packages loaded elsewhere
\PassOptionsToPackage{unicode}{hyperref}
\PassOptionsToPackage{hyphens}{url}
\documentclass[
]{article}
\usepackage{xcolor}
\usepackage[margin=1in]{geometry}
\usepackage{amsmath,amssymb}
\setcounter{secnumdepth}{-\maxdimen} % remove section numbering
\usepackage{iftex}
\ifPDFTeX
  \usepackage[T1]{fontenc}
  \usepackage[utf8]{inputenc}
  \usepackage{textcomp} % provide euro and other symbols
\else % if luatex or xetex
  \usepackage{unicode-math} % this also loads fontspec
  \defaultfontfeatures{Scale=MatchLowercase}
  \defaultfontfeatures[\rmfamily]{Ligatures=TeX,Scale=1}
\fi
\usepackage{lmodern}
\ifPDFTeX\else
  % xetex/luatex font selection
\fi
% Use upquote if available, for straight quotes in verbatim environments
\IfFileExists{upquote.sty}{\usepackage{upquote}}{}
\IfFileExists{microtype.sty}{% use microtype if available
  \usepackage[]{microtype}
  \UseMicrotypeSet[protrusion]{basicmath} % disable protrusion for tt fonts
}{}
\makeatletter
\@ifundefined{KOMAClassName}{% if non-KOMA class
  \IfFileExists{parskip.sty}{%
    \usepackage{parskip}
  }{% else
    \setlength{\parindent}{0pt}
    \setlength{\parskip}{6pt plus 2pt minus 1pt}}
}{% if KOMA class
  \KOMAoptions{parskip=half}}
\makeatother
\usepackage{color}
\usepackage{fancyvrb}
\newcommand{\VerbBar}{|}
\newcommand{\VERB}{\Verb[commandchars=\\\{\}]}
\DefineVerbatimEnvironment{Highlighting}{Verbatim}{commandchars=\\\{\}}
% Add ',fontsize=\small' for more characters per line
\usepackage{framed}
\definecolor{shadecolor}{RGB}{248,248,248}
\newenvironment{Shaded}{\begin{snugshade}}{\end{snugshade}}
\newcommand{\AlertTok}[1]{\textcolor[rgb]{0.94,0.16,0.16}{#1}}
\newcommand{\AnnotationTok}[1]{\textcolor[rgb]{0.56,0.35,0.01}{\textbf{\textit{#1}}}}
\newcommand{\AttributeTok}[1]{\textcolor[rgb]{0.13,0.29,0.53}{#1}}
\newcommand{\BaseNTok}[1]{\textcolor[rgb]{0.00,0.00,0.81}{#1}}
\newcommand{\BuiltInTok}[1]{#1}
\newcommand{\CharTok}[1]{\textcolor[rgb]{0.31,0.60,0.02}{#1}}
\newcommand{\CommentTok}[1]{\textcolor[rgb]{0.56,0.35,0.01}{\textit{#1}}}
\newcommand{\CommentVarTok}[1]{\textcolor[rgb]{0.56,0.35,0.01}{\textbf{\textit{#1}}}}
\newcommand{\ConstantTok}[1]{\textcolor[rgb]{0.56,0.35,0.01}{#1}}
\newcommand{\ControlFlowTok}[1]{\textcolor[rgb]{0.13,0.29,0.53}{\textbf{#1}}}
\newcommand{\DataTypeTok}[1]{\textcolor[rgb]{0.13,0.29,0.53}{#1}}
\newcommand{\DecValTok}[1]{\textcolor[rgb]{0.00,0.00,0.81}{#1}}
\newcommand{\DocumentationTok}[1]{\textcolor[rgb]{0.56,0.35,0.01}{\textbf{\textit{#1}}}}
\newcommand{\ErrorTok}[1]{\textcolor[rgb]{0.64,0.00,0.00}{\textbf{#1}}}
\newcommand{\ExtensionTok}[1]{#1}
\newcommand{\FloatTok}[1]{\textcolor[rgb]{0.00,0.00,0.81}{#1}}
\newcommand{\FunctionTok}[1]{\textcolor[rgb]{0.13,0.29,0.53}{\textbf{#1}}}
\newcommand{\ImportTok}[1]{#1}
\newcommand{\InformationTok}[1]{\textcolor[rgb]{0.56,0.35,0.01}{\textbf{\textit{#1}}}}
\newcommand{\KeywordTok}[1]{\textcolor[rgb]{0.13,0.29,0.53}{\textbf{#1}}}
\newcommand{\NormalTok}[1]{#1}
\newcommand{\OperatorTok}[1]{\textcolor[rgb]{0.81,0.36,0.00}{\textbf{#1}}}
\newcommand{\OtherTok}[1]{\textcolor[rgb]{0.56,0.35,0.01}{#1}}
\newcommand{\PreprocessorTok}[1]{\textcolor[rgb]{0.56,0.35,0.01}{\textit{#1}}}
\newcommand{\RegionMarkerTok}[1]{#1}
\newcommand{\SpecialCharTok}[1]{\textcolor[rgb]{0.81,0.36,0.00}{\textbf{#1}}}
\newcommand{\SpecialStringTok}[1]{\textcolor[rgb]{0.31,0.60,0.02}{#1}}
\newcommand{\StringTok}[1]{\textcolor[rgb]{0.31,0.60,0.02}{#1}}
\newcommand{\VariableTok}[1]{\textcolor[rgb]{0.00,0.00,0.00}{#1}}
\newcommand{\VerbatimStringTok}[1]{\textcolor[rgb]{0.31,0.60,0.02}{#1}}
\newcommand{\WarningTok}[1]{\textcolor[rgb]{0.56,0.35,0.01}{\textbf{\textit{#1}}}}
\usepackage{graphicx}
\makeatletter
\newsavebox\pandoc@box
\newcommand*\pandocbounded[1]{% scales image to fit in text height/width
  \sbox\pandoc@box{#1}%
  \Gscale@div\@tempa{\textheight}{\dimexpr\ht\pandoc@box+\dp\pandoc@box\relax}%
  \Gscale@div\@tempb{\linewidth}{\wd\pandoc@box}%
  \ifdim\@tempb\p@<\@tempa\p@\let\@tempa\@tempb\fi% select the smaller of both
  \ifdim\@tempa\p@<\p@\scalebox{\@tempa}{\usebox\pandoc@box}%
  \else\usebox{\pandoc@box}%
  \fi%
}
% Set default figure placement to htbp
\def\fps@figure{htbp}
\makeatother
\setlength{\emergencystretch}{3em} % prevent overfull lines
\providecommand{\tightlist}{%
  \setlength{\itemsep}{0pt}\setlength{\parskip}{0pt}}
\usepackage{bookmark}
\IfFileExists{xurl.sty}{\usepackage{xurl}}{} % add URL line breaks if available
\urlstyle{same}
\hypersetup{
  pdftitle={lab\_25\_md},
  hidelinks,
  pdfcreator={LaTeX via pandoc}}

\title{lab\_25\_md}
\author{}
\date{\vspace{-2.5em}2025-10-05}

\begin{document}
\maketitle

Hecho con gusto por Carla Carolina Pérez Hernández (UAEH) Alumno: Luis
Armando González Arellano

LABORATORIO - Tidy data -datos ordenados- PARTE 1,2 y 3.

\subsection{Objetivo: Introducción práctica a los datos ordenados (o
tidy data) y a las herramientas que provee el paquete
tidyr.}\label{objetivo-introducciuxf3n-pruxe1ctica-a-los-datos-ordenados-o-tidy-data-y-a-las-herramientas-que-provee-el-paquete-tidyr.}

En este ejercicio vamos a: 1. Cargar datos (tibbles) 2. Converir
nuestros tiblles en dataframes (para su exportación) 3. Exportar
dataframes originales 4. Pivotar tabla 4a 5. Exportar resutltado (TABLA
PIVOTANTE) 6. Separar y unir tablas

\#LABORATORIO: Tidy data (datos ordenados) 1 \#
\#\#\#\#\#\#\#\#\#\#\#\#\#\#\#\#\#\#\#\#\#\#\#\#\#\#\#\#\#\#\#\#\#\#\#\#\#\#\#\#\#\#\#\#\#
\#Prerrequisitos \#instalar paquete tidyverse
install.packages(``tidyverse'')

\#instalar paquete de datos

install.packages(``remotes'')

\begin{Shaded}
\begin{Highlighting}[]
\NormalTok{remotes}\SpecialCharTok{::}\FunctionTok{install\_github}\NormalTok{(}\StringTok{"cienciadedatos/datos"}\NormalTok{)}
\end{Highlighting}
\end{Shaded}

\begin{verbatim}
## Skipping install of 'datos' from a github remote, the SHA1 (7f69393d) has not changed since last install.
##   Use `force = TRUE` to force installation
\end{verbatim}

install.packages(``datos'')

\#Cargar paquete tidyverse

\begin{Shaded}
\begin{Highlighting}[]
\FunctionTok{library}\NormalTok{(tidyverse)}
\end{Highlighting}
\end{Shaded}

\begin{verbatim}
## -- Attaching core tidyverse packages ------------------------ tidyverse 2.0.0 --
## v dplyr     1.1.4     v readr     2.1.5
## v forcats   1.0.1     v stringr   1.5.2
## v ggplot2   4.0.0     v tibble    3.3.0
## v lubridate 1.9.4     v tidyr     1.3.1
## v purrr     1.1.0     
## -- Conflicts ------------------------------------------ tidyverse_conflicts() --
## x dplyr::filter() masks stats::filter()
## x dplyr::lag()    masks stats::lag()
## i Use the conflicted package (<http://conflicted.r-lib.org/>) to force all conflicts to become errors
\end{verbatim}

\#Cargar paquete de datos

\begin{Shaded}
\begin{Highlighting}[]
\FunctionTok{library}\NormalTok{(datos)}
\end{Highlighting}
\end{Shaded}

\#tabla 1 hasta tabla 4b \#ver datos como tibble

\begin{Shaded}
\begin{Highlighting}[]
\NormalTok{datos}\SpecialCharTok{::}\NormalTok{ tabla1}
\end{Highlighting}
\end{Shaded}

\begin{verbatim}
## # A tibble: 6 x 4
##   pais        anio  casos  poblacion
##   <chr>      <dbl>  <dbl>      <dbl>
## 1 Afganistán  1999    745   19987071
## 2 Afganistán  2000   2666   20595360
## 3 Brasil      1999  37737  172006362
## 4 Brasil      2000  80488  174504898
## 5 China       1999 212258 1272915272
## 6 China       2000 213766 1280428583
\end{verbatim}

\begin{Shaded}
\begin{Highlighting}[]
\NormalTok{datos}\SpecialCharTok{::}\NormalTok{ tabla2}
\end{Highlighting}
\end{Shaded}

\begin{verbatim}
## # A tibble: 12 x 4
##    pais        anio tipo          cuenta
##    <chr>      <dbl> <chr>          <dbl>
##  1 Afganistán  1999 casos            745
##  2 Afganistán  1999 población   19987071
##  3 Afganistán  2000 casos           2666
##  4 Afganistán  2000 población   20595360
##  5 Brasil      1999 casos          37737
##  6 Brasil      1999 población  172006362
##  7 Brasil      2000 casos          80488
##  8 Brasil      2000 población  174504898
##  9 China       1999 casos         212258
## 10 China       1999 población 1272915272
## 11 China       2000 casos         213766
## 12 China       2000 población 1280428583
\end{verbatim}

\begin{Shaded}
\begin{Highlighting}[]
\NormalTok{datos}\SpecialCharTok{::}\NormalTok{ tabla3}
\end{Highlighting}
\end{Shaded}

\begin{verbatim}
## # A tibble: 6 x 3
##   pais        anio tasa             
##   <chr>      <dbl> <chr>            
## 1 Afganistán  1999 745/19987071     
## 2 Afganistán  2000 2666/20595360    
## 3 Brasil      1999 37737/172006362  
## 4 Brasil      2000 80488/174504898  
## 5 China       1999 212258/1272915272
## 6 China       2000 213766/1280428583
\end{verbatim}

\begin{Shaded}
\begin{Highlighting}[]
\NormalTok{datos}\SpecialCharTok{::}\NormalTok{ tabla4a}
\end{Highlighting}
\end{Shaded}

\begin{verbatim}
## # A tibble: 3 x 3
##   pais       `1999` `2000`
##   <chr>       <dbl>  <dbl>
## 1 Afganistán    745   2666
## 2 Brasil      37737  80488
## 3 China      212258 213766
\end{verbatim}

\begin{Shaded}
\begin{Highlighting}[]
\NormalTok{datos}\SpecialCharTok{::}\NormalTok{ tabla4b}
\end{Highlighting}
\end{Shaded}

\begin{verbatim}
## # A tibble: 3 x 3
##   pais           `1999`     `2000`
##   <chr>           <dbl>      <dbl>
## 1 Afganistán   19987071   20595360
## 2 Brasil      172006362  174504898
## 3 China      1272915272 1280428583
\end{verbatim}

\#ver datos como dataframe

\begin{Shaded}
\begin{Highlighting}[]
\NormalTok{df1 }\OtherTok{\textless{}{-}}  \FunctionTok{data\_frame}\NormalTok{(tabla1)}
\end{Highlighting}
\end{Shaded}

\begin{verbatim}
## Warning: `data_frame()` was deprecated in tibble 1.1.0.
## i Please use `tibble()` instead.
## This warning is displayed once every 8 hours.
## Call `lifecycle::last_lifecycle_warnings()` to see where this warning was
## generated.
\end{verbatim}

\begin{Shaded}
\begin{Highlighting}[]
\NormalTok{df2 }\OtherTok{\textless{}{-}}  \FunctionTok{data\_frame}\NormalTok{(tabla2)}
\NormalTok{df3 }\OtherTok{\textless{}{-}}  \FunctionTok{data\_frame}\NormalTok{(tabla3)}
\NormalTok{df4a }\OtherTok{\textless{}{-}}  \FunctionTok{data\_frame}\NormalTok{(tabla4a)}
\NormalTok{df4b }\OtherTok{\textless{}{-}}  \FunctionTok{data\_frame}\NormalTok{(tabla4b)}
\end{Highlighting}
\end{Shaded}

\#exportar los dataframes originales

\begin{Shaded}
\begin{Highlighting}[]
\FunctionTok{write.csv}\NormalTok{(df1, }\AttributeTok{file=} \StringTok{"df1.csv"}\NormalTok{)}
\FunctionTok{write.csv}\NormalTok{(df2, }\AttributeTok{file=} \StringTok{"df2.csv"}\NormalTok{)}
\FunctionTok{write.csv}\NormalTok{(df3, }\AttributeTok{file=} \StringTok{"df3.csv"}\NormalTok{)}
\FunctionTok{write.csv}\NormalTok{(df4a, }\AttributeTok{file=} \StringTok{"df4a.csv"}\NormalTok{)}
\FunctionTok{write.csv}\NormalTok{(df4b, }\AttributeTok{file=} \StringTok{"df4b.csv"}\NormalTok{)}
\end{Highlighting}
\end{Shaded}

vignette(``tibble'') \#explicación de tibble

\section{Ordenar datos con la tabla4a
(PIVOTAR)}\label{ordenar-datos-con-la-tabla4a-pivotar}

\begin{Shaded}
\begin{Highlighting}[]
\NormalTok{t4a\_PIVOTANTE}\OtherTok{=}\NormalTok{ tabla4a }\SpecialCharTok{\%\textgreater{}\%} 
  \FunctionTok{pivot\_longer}\NormalTok{(}\AttributeTok{cols =} \FunctionTok{c}\NormalTok{(}\StringTok{"1999"}\NormalTok{, }\StringTok{"2000"}\NormalTok{), }\AttributeTok{names\_to =} \StringTok{"anio"}\NormalTok{, }\AttributeTok{values\_to =}\StringTok{"casos"}\NormalTok{) }
\end{Highlighting}
\end{Shaded}

\section{Exportar resultado: tabla
ordenada}\label{exportar-resultado-tabla-ordenada}

\begin{Shaded}
\begin{Highlighting}[]
\FunctionTok{write.csv}\NormalTok{(t4a\_PIVOTANTE, }\AttributeTok{file =} \StringTok{"t4a\_PIVOTANTE.csv"}\NormalTok{)}
\end{Highlighting}
\end{Shaded}


\end{document}
